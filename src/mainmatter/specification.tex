\chapter{Specification}
\label{Chapter:Specification}
  To obtain a complete and practical understanding of IoT, the software implementation accompanying this paper will comprise of two distinct components: a generic IoT application framework and a demonstration application. \citet{interoperability:2015} suggest that developers would benefit from a generic architecture which provides a common set of tools, and that is the aim of the framework component. The second component has two main purposes. First of all, it will build on top of this framework with the aim of validating design decisions. The second purpose is to demonstrate the opportunities present in IoT. This chapter will define the requirements for both of these components.

  To help reference these two software components throughout the remained of this paper, they have been given product names. Haar is a cold sea fog which forms on the east coast of Scotland and is a fitting metaphor for IoT being all around us. It is also a nod towards the Fog Computing paradigm, and towards Robert Gordon University's location in Aberdeen. The generic framework will be called the Haar Engine and the demonstration application will be named Haar.

  \section{Functional Requirements}
    \subsection{Framework}
      The target users of the Haar Engine are designers and developers of IoT services and applications. The general aim of a framework is to encapsulate design decisions into a logical and elegant toolkit. The Haar Engine aims to perform the same task by encapsulating the knowledge which has been collated in Chapter \ref{Chapter:Background}. The following subsection specifies the main requirements.

      \subsubsection{User and Device Management}
        Users and devices are two overlapping components of an IoT framework. Nitrogen.io is an existing IoT platform and refers to users and devices as \emph{principals} of the system. The documentation states that principals are a unit of authentication and authorisation. Since users and devices of the Haar Engine share a number of overlapping requirements, this project will adopt the same terminology. The following requirements have been derived from analysis of existing platforms and the requirements of use cases.

        \begin{enumerate}
          \item The framework shall model users
          \begin{enumerate}
            \item A user shall be identified by a unique username
            \item A user shall have an associated profile including: full name, email address
            \item Users shall be assigned a privilege level: standard or administrator
          \end{enumerate}
          \item The framework shall authenticate genuine users only

          \item The framework shall model connected IoT devices
          \begin{enumerate}
            \item A device shall be identified by a unique identification number
            \item A device shall be designated as either an input (sensor) or output (actuator)
          \end{enumerate}
          \item The framework shall authenticate genuine devices only

          \item The framework shall authorise and facilitate the management of devices
          \begin{enumerate}
            \item A device will have one owner
            \item Access to device data and configuration options shall be restricted to the device owner by default
            \item The device owner shall be able to make device data public
            \item A device owner shall be able to grant access permission to specified users
          \end{enumerate}
        \end{enumerate}

      \subsubsection{Device Connectivity}
        A main aspect of the IoT concept is the interconnection of devices. The Haar Engine is responsible for managing the connectivity between devices and itself. The issue surrounding bidirectional communication was described in Section \ref{bidirectioncomms}, which should be addressed by this framework.

        \begin{enumerate}
          \item The framework shall establish a connection with end devices
          \begin{enumerate}
            \item Devices shall be addressable and identifiable whilst the connection is open
            \item The connection shall remain open under normal operating conditions
            \item Under error conditions, the connection shall be reestablished
          \end{enumerate}
          \item The connection shall be bidirectional
          \begin{enumerate}
            \item Input devices shall be able to transmit data packets to the framework
            \item The framework shall be able to transmit data packages to output devices
            \item The connection shall be tolerant of network obstacles such as firewalls and Network Address Translation
          \end{enumerate}
        \end{enumerate}

      \subsubsection{Data Storage}
        IoT applications will collect a large volume of data. The data collected by these applications will vary in structure and type, so the Haar Engine must be tolerant of this. The developers potentially using this framework will have specific domain knowledge and so should be empowered to tailor the system for their needs.

        \begin{enumerate}
          \item The framework shall store data received by devices
          \item The framework shall provide an interface for using different database types
          \begin{enumerate}
            \item The framework shall provide a default `out of the box' database connector
            \item Application developers shall be able to create their own database connectors
          \end{enumerate}
        \end{enumerate}

      \subsubsection{Data Access API}
        The Haar Engine must be able to provide access to stored data. Primarily this access will be for use in its own application, however 3rd-party access would enable interoperability between external services and applications.

        \begin{enumerate}
          \item The framework shall provide structured access to stored data
          \begin{enumerate}
            \item Access shall be granted to authenticated and authorised users only
          \end{enumerate}
          \item Application developers shall be able to modify data access
          \begin{enumerate}
            \item Application developers shall be able to add custom data searches
            \item Application developers shall be able to add custom data filters
          \end{enumerate}
        \end{enumerate}

      \subsubsection{Real-Time Event API}
        The ability to react to data in real-time enables novel use cases. As with the Data Access API, the primary purpose is to satisfy the needs of a specific application. However, 3rd-party access to this API would allow for further real-time integrations.

        \begin{enumerate}
          \item The framework shall provide a real-time event API
          \begin{enumerate}
            \item Client applications shall be able to listen for device events
            \item The real-time API shall leverage bidirectional connectivity with devices
          \end{enumerate}
          \item The API shall provide real-time access to data received from devices
          \item The real-time API shall enable developers to create application rules and triggers
        \end{enumerate}

    \subsection{Demonstration Application}
      Haar is the second software component for this project. The aim of Haar is to validate the Haar Engine in practice, as well as to demonstrate the opportunities present in the IoT concept. This will be achieved through the implementation of an abitrary web-based application which sufficiently covers key features.

      \subsubsection{Devices}
        The IoT concept enables the interconnection of devices so this is a logic requirement to begin with. Haar should comprise of at least two seperate local device networks to demonstrate global interactivity. These local networks should themselves comprise of a number of devices with enough variety to validate Haar Engine use cases.

        \begin{enumerate}
          \item The application shall comprise of two local area device networks
          \item Devices of these networks shall be capable of establishing a connection between themselves and the web application
          \item The device networks shall comprise of two sensor devices
          \begin{enumerate}
            \item The device network shall contain a sensor device which measures a single data point, such as temperature
            \item The device network shall contain a sensor device which measures more complex data, such as a vector of wind direction and speed 
          \end{enumerate}
          \item The device networks shall comprise of a single actuator device
          \begin{enumerate}
            \item The actuator device shall be flexible enough to represent a variety of data types
          \end{enumerate} 
        \end{enumerate}

      \subsubsection{Dashboard}
        The web-based dashboard will be the main interface between a user and their devices.

        \begin{enumerate}
          \item The dashboard shall be accessible on the World Wide Web
          \item The dashboard shall authenticate users with a username and password
          \item The dashboard shall list profile details
          \item The user shall be able to manage their devices
          \begin{enumerate}
            \item The dashboard shall list devices owned by the user
            \item The user shall be able to add a new device
            \item The dashboard shall show whether devices are connected or not
            \item The user shall be able to transfer ownership of the device
            \item The user shall be able to grant access permissions to other users
          \end{enumerate}
          \item The dashboard shall enable users manage application rules
          \begin{enumerate}
            \item The dashboard shall list existing rules
            \item The user shall be able to create new rules
            \item An actuator device shall be able to react to sensed data
            \item User devices shall be able to integrate with 3rd-party services [See 3rd-Party Integration overleaf]
          \end{enumerate}
          \item The dashboard shall display data for a given device [see Data Visualisation below]
          \item The dashboard shall be able to trigger actuators with virtual controls
        \end{enumerate}

      \subsubsection{Data Visualisation}
        There is an opportunity to develop a rich user interface for the dashboard. Since so much of IoT relies on data, it would be appropriate to develop example data visualisation tools for it.

        \begin{enumerate}
          \item A section of the dashboard shall be dedicated to data visualisation
          \item A generic visualisation tool shall plot temporal data as a graph
          \begin{enumerate}
            \item The date range shall be configurable
            \item Multiple data sources shall be comparable on one graph interface
            \item The graph shall leverage the real-time event API to update when new data is received
          \end{enumerate}
          \item Specialised visualisation tools shall reflect specific device types. For example, a virtual thermometer would visualise a temperature
          \begin{enumerate}
            \item Specialised visualisation tools shall leverage the real-time event API to update when new data is received
          \end{enumerate}
        \end{enumerate}

      \subsubsection{3rd-Party Integration}
        Interoperability between IoT services is described as a challenge in Section \ref{challenges}. Haar should therefore demonstrate an ability to integrate with third-party services.

        \begin{enumerate}
          \item The application shall provide pre-defined 3rd-party integrations
          \item The integrations shall be listed as 3rd-party rules
          \item One integration shall demonstrate reaction to a 3rd-party event
          \item One integration shall trigger a 3rd-party service in response to new data
        \end{enumerate}  

  \section{Non-functional Requirements}
    In addition to the features required for Haar and the Haar Engine, there are a number of additional considerations. Since both of these software components are destined for use in the same system, this section will detail their non-function requirements collectively.

    \subsection{Constraints}
      \subsubsection{Network Environment}
        The potential operating environments in which IoT devices could be implemented vary from domestic household appliances to heavy industrial applications. These networks will all have different configurations, capabilities and customisability. In order to reduce deployment issues, the software components must operate in a common network scenario. A typical home network will be used as a baseline, where only common firewall ports will be allowed.

      \subsubsection{Public IP Address}
        IPv6 is the successor to IPv4 and will allow every connected device to have a unique, globally addressable IP address. While this is the direction the industry is heading in, many businesses and Internet Service Providers are not prepared its adoption. Network Address Translation (NAT) is still commonly deployed and prevents devices on these networks from being publically addressable. It should therefore be assumed that end devices will not have a public IP address.

      \subsubsection{Initial Setup}
        In a domestic situation, consumers would be expected to connect products to their internet connection by themselves. This is an important user experience to get right. While this project acknowledges the importance of this, it will not be considered a requirement for Haar so more effort can be afforded to developing a robust application.

      \subsubsection{Sensor Nodes}
        To demonstrate the various framework use cases, the application should connect a variety of sensor types. To ensure this project remains achieveable, sensor nodes will be limited to those whose data can be represented by a single data interchange format like XML or JSON. Binary data such as images or audio will not be considered for this project.

    \subsection{Performance Requirements}
      Since this project is a research endeavor, it neither requires nor promises any performance benchmarks. In saying that, the implemented product should meet general performance expectations, especially in regards to any web-based user interfaces.

    \subsection{Security Requirements}
      This particular software product will not store sensitive personal details, nor will it need to satisfy criteria for professional accreditation. However security is an important topic in IoT and so real-world security measures should be implemented. In particular, any device or application connectivity should be implemented over encrypted connections to prevent potential network sniffing. Also any live demonstration versions of this software should be hosted on up-to-date hosting services to prevent any bug exploitations.

    \subsection{Software Quality Attributes}
      \subsubsection{Extensibility}
        The application domain of an IoT project could be in virtually any industry. It is important that this software product is extensible, meaning that the core functionality can be extended to cover specific use-cases.

      \subsubsection{Portability}
        The server infrastructure supporting a generic platform such as this could vary between organisations and use-cases. For instance it may be beneficial to use a Platform as a Service such as Heroku or conversely a range of dedicated servers may be in use. This project should be platform agnostic. 

      \subsubsection{Scalability}
        It is difficult to predict the exact number of sensor nodes and the data which they will produce. The software components should be scalable, either horizontally and/or vertically.

      \subsubsection{Testability}
        It is important that core system functions execute as designed. This project should therefore be testable through assertions where possible and appropriate.

      \subsubsection{Robustness}
        With many devices relying on a system for data and connectivity, system stability is an important focus. The framework should strive to be as fault tolerant as possible, however this is the sum of many parts. Reliable hardware infrastructure and high quality tested code all play a part.

      \subsubsection{Reusability}
        Of course a main feature of a generic framework is the ability to be reused. Two different organisations with different use-cases should both be able to use the framework without issue.