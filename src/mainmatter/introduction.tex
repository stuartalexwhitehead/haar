\chapter{Introduction}
  \section{Project Motivation}
  \section{Aims and Objectives}
    % Security as a mindset
  \section{Report Structure}

  \section{Legal, Social, Ethical and Professional Considerations}
  \label{chapter:profession-considerations}
    There are a number of legal, social, ethical and professional issues to consider at all points throughout this project. It is my responsibility and my responsibility alone to ensure that these considerations are upheld.

    The ultimate aim of this project is to satisfy requirements for my Computer Science course---a course which is accredited by the British Computer Society. The BCS maintain a Code of Conduct \citep{bcs-coc} so it is logical that it should inform the responsibilities of this project. More specifically, the code covers four main areas: public interest; professional competence and integrity; duty to relevant authority and duty to the profession.

    \subsection{Public Interest}
      The public interest clause ensures that the legitimate rights of the public and other third parties are respected. With respect to the project focus on the Internet of Things, point 1 (a) of the BCS Code of Coduct should be considered: ``have due regard for public health, privacy, security and wellbeing of others and the environment.''.

      Internet of Things devices have the potential to harvest large volumes of private data about people and places. Health-monitoring devices in particular have the specific capability to measure sensitive medical information; devices like this will not be considered for this project. The devices which have been developed (ambient temperature, colour and gyroscopic motion) are intentionally generic and have no particular privacy requirements. Any users who have been invited to use the devices will be explained as to their purpose.

      The developed application has the capability to store basic identifiable information. In particular, a user profile includes: first and last names; username; password and email address. For the purposes of testing and demonstration, the information database will be seeded with fake data so no user information is at risk. Nonetheless, steps have been taken in order to protect data as far as feasibly possible, including: hardened web hosting with strict a firewall policy; Docker-based deployment environments; random and suitably long alphanumeric passwords for root/admin users.

      Consideration has also been given towards environmental responsibility. Any hardware waste has been disposed of in a responsible manner. For example, spent batteries have been disposed of in a battery bin at a local shop. Hardware devices (wired or battery powered) are switched off when not in use in order to conserve electical power. Documents have only been printed when needed in order to prevent paper wastage. 

    \subsection{Professional Competence and Integrity}
      The professional competence clause ensures that an individual exhibits a true portrayal of their knowledge and capabilities. Effort will be made in every case to understand a topic to the required level. Point 2 (e) is particularly important for the academic nature of this project: ``respect and value alternative viewpoints and, seek, accept and offer honest criticisms of work.''. Opinions and criticisms of this project are welcomed and responses will be made where approriate. 

    \subsection{Duty to Relevant Authority}
      The `Duty to Relevant Authority' clause ensures that due care is exercised with regards to any applicable legislation. First of all, as author I accept responsibility for all work detailed as part of this project. This project will also take all foreseeable measures in order to avoid a conflict of interest with the relevant authority. In the event where information is requested, an approriate response and disclosure will be made.

    \subsection{Duty to the Profession}
      The final clause of the BCS Code of Conduct ensures that the project contributes to and upholds the professional reputation of the computing industry. Point 4 (d) states: ``act with integrity and respect in your professional relationships''. This should be considered with particular attention to software copyright and licensing conditions. Full credit will be given to the authors of software included within the output of this project. Where required, the licensing terms of software (especially Open Source Software) will be upheld.
