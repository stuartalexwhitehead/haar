\chapter{Introduction}
  The Internet of Things (IoT) is a computing term which has entered the common lexicon. The general public have been introduced to the idea of Internet-connected devices through frivolous products like self-replenishing fridges or pizza-order buttons \citep{dominos-order}. But the potential of Internet of Things devices reach far beyond these marketing gimmicks.

  IoT devices can and will generate value through large-scale deployments. Their small size, connectivity and energy efficiency will enable a new generation of services. These services will have the capability to measure our world on a scale and with a precision which has never before been possible. It is hoped that this paper will open your mind to the technology and endless possibilities evident with wirelessly connected sensor networks.

  \section{Project Motivation}
    My curiosity in IoT has been building for many years. It is difficult to detemine exactly when I came across the term, but my current understanding of it and its potential can be traced back to the Mobile User Experience (MEX) conference in London. In 2013 I was awarded a scholarship to attend, and also returned for three subsequent events as a volunteer. The conference itself is small, focused and addresses strategies and design methods for next generation digital experiences. One recurring topic in presentations and discussions was the Internet of Things.

    There are two main reasons why this topic was selected as the focus of my honours project. The first reason is that IoT is a very active area of research. Academic institutions and manufacturers alike are investing heavily in order to develop truly valuable solutions. This is also reflected in the 2015 Gartner Hype Cycle. Gartner's market research places IoT at the `Peak of Inflated Expectations' \citep{gartner-hype-cycle}, a title given to technologies which have had widespread publicity but are not ready for mainstream adoption.

    The second reason for choosing IoT is that it is a genuinely interesting area. IoT is multidisciplinary and can be found at the intersection of computing, design, physics and engineering. I have an active interest in all of these areas so this has helped provide motivation throughout the project.

  \section{Aims and Objectives}
    A number of high-level aims and objectives were defined to orientate the project in a purposeful direction. These were left intentionally broad so that as much of the IoT landscape could be investigated as possible. By keeping the aims broad, it has also allowed background research to guide the project journey. The aims represent the purpose of this project whereas objectives define the main tasks to undertake.

    \subsubsection{Aims}
    Three aims were defined for this project:
    \begin{enumerate}
      \item To satisfy the requirements of the BSc (Hons) Computer Science course
      \item To satisfy my curiosity of the research area
      \item To demonstrate skills to potential employers
    \end{enumerate}

    \subsubsection{Objectives}
    These aims were achieved through the completion of four main objectives:
    \begin{enumerate}
      \item Research and synthesise IoT concepts and techniques
      \item Research and evaluate technologies which support IoT
      \item Design and implement an IoT application framework
      \item Design and implement a full-stack IoT application based on the framework
    \end{enumerate}

  \section{Report Structure}
    The structure of this report reflects the software development lifecycle. The contents of each section are described below.

    \begin{description}
      \item[Background] A broad literature review on IoT concepts, practices and technologies.
      \item[Specification] Functional and non-functional requirements of the IoT framework and demonstration application. These are derived from information gathered in the preceeding chapter.
      \item[Design] System architecture and high-level implementation decisions such as programming language and hardware schematics for the IoT framework and demonstration application.
      \item[Development Practices] A discussion of professional development practices used to structure the implementation.
      \item[Implemention] Specific implementation details for each of the software and hardware components.
      \item[Evaluation] A discussion on the effectiveness of the developed software components.
      \item[Conclusion and Future Work] A reflection on the IoT landscape, a critical appraisal and potential work leading on from this project.

    \end{description}

  \section{Legal, Social, Ethical and Professional Considerations}
  \label{chapter:profession-considerations}
    There are a number of legal, social, ethical and professional issues to consider at all points throughout this project. It is my responsibility and my responsibility alone to ensure that these considerations are upheld.

    The ultimate aim of this project is to satisfy requirements for my Computer Science course---a course which is accredited by the British Computer Society. The BCS maintain a Code of Conduct \citep{bcs-coc} so it is logical that it should inform the responsibilities of this project. More specifically, the code covers four main areas: public interest; professional competence and integrity; duty to relevant authority and duty to the profession.

    \subsection{Public Interest}
      The public interest clause ensures that the legitimate rights of the public and other third parties are respected. With respect to the project focus on the Internet of Things, point 1 (a) of the BCS Code of Coduct should be considered: ``have due regard for public health, privacy, security and wellbeing of others and the environment.''.

      Internet of Things devices have the potential to harvest large volumes of private data about people and places. Health-monitoring devices in particular have the specific capability to measure sensitive medical information; devices like this will not be considered for this project. The devices which have been developed (ambient temperature, colour and gyroscopic motion) are intentionally generic and have no particular privacy requirements. Any users who have been invited to use the devices will be explained as to their purpose.

      The developed application has the capability to store basic identifiable information. In particular, a user profile includes: first and last names; username; password and email address. For the purposes of testing and demonstration, the information database will be seeded with fake data so no user information is at risk. Nonetheless, steps have been taken in order to protect data as far as feasibly possible, including: hardened web hosting with strict a firewall policy; Docker-based deployment environments; random and suitably long alphanumeric passwords for root/admin users.

      Consideration has also been given towards environmental responsibility. Any hardware waste has been disposed of in a responsible manner. For example, spent batteries have been disposed of in a battery bin at a local shop. Hardware devices (wired or battery powered) are switched off when not in use in order to conserve electical power. Documents have only been printed when needed in order to prevent paper wastage. 

    \subsection{Professional Competence and Integrity}
      The professional competence clause ensures that an individual exhibits a true portrayal of their knowledge and capabilities. Effort will be made in every case to understand a topic to the required level. Point 2 (e) is particularly important for the academic nature of this project: ``respect and value alternative viewpoints and, seek, accept and offer honest criticisms of work.''. Opinions and criticisms of this project are welcomed and responses will be made where approriate. 

    \subsection{Duty to Relevant Authority}
      The `Duty to Relevant Authority' clause ensures that due care is exercised with regards to any applicable legislation. First of all, as author I accept responsibility for all work detailed as part of this project. This project will also take all foreseeable measures in order to avoid a conflict of interest with the relevant authority. In the event where information is requested, an approriate response and disclosure will be made.

    \subsection{Duty to the Profession}
      The final clause of the BCS Code of Conduct ensures that the project contributes to and upholds the professional reputation of the computing industry. Point 4 (d) states: ``act with integrity and respect in your professional relationships''. This should be considered with particular attention to software copyright and licensing conditions. Full credit will be given to the authors of software included within the output of this project. Where required, the licensing terms of software (especially Open Source Software) will be upheld.
